\section{Introduction}

% WTP: What are simulation based models, what is bayesian inference in them.  Why is it necessary. 
For many phenomena we lack succinct equations governing their behaviour but nevertheless understand the causal mechanisms by which they evolve.
Such phenomena can be simulated, which means to execute a program and associate states which arise in its execution with entities in the domain.
% as an algorithm which solves the equations which govern it. Values, states, and transitions in the execution of the algorithm correspond to entities in the domain.
This execution generally follows the causal direction of the domain in the sense that if $a$ causes $b$, $b$ is computed from $a$ in simulation.
Problems of inference, on the other hand, are typically anti-causal: having observed the result of the simulation we wish to infer the causes.



% WTP: What are likelihood functions, why are they use useful, and why are they intractable
The Bayesian approach to inference is to first quantify all uncertainty with probability distributions, and as a result uniquely specify the posterior distribution given observed values.
% Bayesian inference is a principled framework for inferring probable causes of observed values, but relies on terms which are often intractable.
Most algorithms which draw samples from posterior distributions rely on
% Bayes theorem dictates that the posterior distribution over latent variables given observations is is proportional to the product of the likelihood function and prior.
the likelihood function, which quantifies the extent to which values of latent variables are consistent with observations.
Simulation models explicitly represent the processes which generate data, but leave the likelihood functions implicit. 
The likelihood  difficult to derive from a simulation model and is often intractable to evaluate.
This can occur, for example, when non-latent variables are unobserved, since they must be margainalized out, or when transformed densities are conditioned.
Simulation based models can apply arbitrary, non-injective transformations to their input, and hence induce intractable likelihoods for any or all of these reasons.

A less explored cause of intractability are propositions that are not observations in the conventional sense.
To observe data $x$ as the output of a simulation model $X$ means to condition the model on the proposition $X = x$.
There are several propositions that do not conform to this structure, which we would nevertheless like to condition our models on.
These propositions may impose constraints on latent variables, or even jointly on latent and observations.
For example.
They could also be observational but more abstract than concrete data.
For example,.
All such statements can be enforced through conditioning, but generally the resulting likelihood will be intractable or may not exist,
% Morever, approximate inference algorithm 

% % A less explored cause of intractability occurs when we condition on propositions that are not observations of data.
% For example, if $X$ and $Y$ are normally distributed random variables, then conditioning on $X = Y$ does not permit a density function, let alone a tractable one.
% Conditioning on inequalities such $X > 4$, transformations such $X^2  = 0$, and logical formulas such as  $X = 3 \lor X = 5$ falls outside the domain of virtually all conditional sampling algorithms.
% % As a consequence, conditioning in practice is significantly more limited than its theoretical counterpart.

% The value of probabilistic programming systems hinges on how easily a practitioner can encode domain knowledge into a model. Existing probabilistic programming systems support two main mechanisms for encoding knowledge: implicitly in the generative model, and explicitly by conditioning on observations. There are many contexts, however, for which neither of these two mechanisms are sufficient to encode the knowledge that practitioners have about a process.

% WTP: Implications of the Problem

Conditioning is a mechanism to express declarative knowledge.
It allows us to assert facts we know to be true without specifying the means by which they are true.
% Limiting the propositions one can condition on to those with tractable likelihoods severely restricts the kinds of knowledge that we can express.
For example, consider the problem of modeling the evolution of glucose levels over time~\citep{levine2017offline}.
A scientist could define a generative model that captures prior knowledge of how glucose levels evolve over time -- for example, the model may use latent variables to identify when a person eats and the glycemic loads a person consumes, and encode physiological knowledge of how those meals will affect glucose levels.
The scientist could also condition the model on observations of glucose measurements of a given patient and use existing inference algorithms to sample from the posterior distributions over latent variables.
However, there are various pieces of declarative knowledge the scientist may possess which are difficult or impossible to encode in this form.
For instance, she may know that ``human glucose curves are similar across patients'',   is surprisingly challenging even in the most expressive probabilistic programming systems. 
Rather than observations, these propositions are facts which condition the prior.These propositions are not observations; 

% WTP: Non-obvious benefits of solving the problem:  Prior conditioning.
Declarative knowledge has the potential to serve as the bridge between classical parametric models, and recent trend towards nonparameric or highly parameterized families such as deep neural networks.
For example, inverse graphics attempts to infer the three dimensional scene (geometry, lights, camera, etc) which caused an observed image.
A Bayesian formulation of the problem requires a prior distributions over scenes.
Simple parametric stand little chance of capturing the complexity of real world scenes, prompting.

% The above part needs to be tightened, but the story is good a ton


% WTP: Contribution: inference algorithm that supports conditioning on a wider class of propositions
In this paper we present an algorithm that draws samples from generative models that have been conditioned on predicates belonging to a more general class than observation of data.
Predicates, when used as black boxes, provide only sparse information -- the constraint is satisfied or it is not -- and the subset of satisfying constraints is typically vanishingly small.
Our objective is to support conditioning on predicates on spaces for which a natural metric can be defined.
A metric provides more information a measure of the degree of satisfaction, and allows us.

% WTP: Paper summary
In summary we address the problem of conditioning on declarative knowledge.
In more detail:
\begin{itemize}
\item We formalize simulation models in measure-theoretic probability as random variables defined on a shared probability space (section X), and define conditioning as a concentration of measure.
\item We describe our approach to inference, which softens the hard constraints to admit tractable inference in a broader set of scenarios.
\item  We demonstrate our approach on a number of examples, with experiments on toy data and experiments on medical models by enriching them with declarative knowledge to learn from limited data.
\end{itemize}