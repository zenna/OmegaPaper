% Predicates are the means to express declarative knowledge.
% We address the problem of conditioning probabilistic models on predicates, as a means to express declarative knowledge.
Probabilistic programming languages allow us to express complex predicates but restrict us from conditioning models on most of them.

Models conditioned on predicates rarely have a tractable likelihood and likelihood-free inference methods
only handle predicates which express observations of data.
To address a broader class of predicates, we develop an inference procedure called \emph{predicate exchange}, which softens predicates.
Soft predicates return values in a continuous Boolean algebra and replace the likelihood function in the posterior.
Softening introduces error; soft predicates approximate their hard counterparts.
The degree of error can be increased or decreased, which trades off between  a parameterized approximation error which when increased
To mitigate this trade-off, develop a derivative of replica exchange, a Markov Chain Monte Carlo method originating in statistical physics.
We implement predicate exchange through a nonstandard execution of a simulation based model, and provide a light-weight tool that can be supplanted on top of existing probabilistic programming formalisms. 
We demonstrate the approach on sequence models of health and inverse rendering. 



% 