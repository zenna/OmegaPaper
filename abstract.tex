We develop a likelihood free inference procedure for conditioning a probabilistic model on a predicate.

A predicate is a Boolean valued function which expresses a yes/no question about a domain.

Our contribution, which we call predicate exchange, 
constructs a softened predicate which takes value in the unit interval [0, 1] as opposed to a simply true or false. 

Intuitively, 1 corresponds to true, and a high value (such as 0.999) corresponds to ``nearly true'' as determined by a distance metric.

We define Boolean algebra for soft predicates,  such that they can be negated, conjoined and disjoined arbitrarily.

A softened predicate can serve as a tractable proxy to a likelihood function for approximate posterior inference.

However, to target exact inference, we temper the relaxation by a temperature parameter, and add a accept/reject phase use to replica exchange Markov Chain Mont Carlo, which exchanges states between a sequence of models conditioned on predicates at varying temperatures.

We describe a lightweight implementation of predicate exchange that it provides a language independent layer that can be implemented on top of existingn modeling formalisms.

---
Conditioning a model on a predicate builds new model. Examples
include ...

To sample from these models, we need likelihood free inference
for predicates. 

However traditional approaches to likelihood 
free inference for predicate tradeoff approximation quality
with computational tractability.

We develop a likelihood free inference procedure 
called predicate exchange. Predicate exchange 
transforms a predicate so that it returns 
a value in a soft Boolean algebra:the unit
interval with continuous logical connections.

The predicates in soft Boolean algebra serves as a proxy likelihood
function in inference.



% 