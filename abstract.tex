We develop a likelihood free inference procedure for conditioning a probabilistic model on a predicate.
A predicate is a Boolean valued function which expresses a yes/no question about a domain.
When a model is conditioned on a predicate, the model is revised such that the predicate becomes a true proposition.
An observation is a special kind of predicate, that is true if and only if a variable is equal to a value.
Predicates can express propositions other than the observation of data.
For instance, they can assert constraints on latent variables, rather than observables.
They can express observations that are more coarse than variables in the model.
Furthermore, predicates can be combined together in complex logical combinations.
Unfortunately, conditioning a model on a predicates remains severely limited due to the fact that the corresponding likelihood is often intractable to compute, if it is available at all.
Our contribution, which we call predicate exchange, 
constructs a softened predicate which takes value in the unit interval [0, 1] as opposed to simply true or false. Intuitively, 1 corresponds to true, and a high value (such as 0.999) corresponds to "nearly true" as determined by a distance metric.
We define Boolean algebra for soft predicates,  such that they can be negated, conjoined and disjoined arbitrarily.
A softened predicate can serve as a tractable proxy to a likelihood function for approximate posterior inference.
However, we target exact inference by tempering the relaxation by a temperature parameter, and use replica exchange Markov Chain Mont Carlo, which exchanges states between a sequence of models conditioned on predicates at varying temperatures.
In doing so, we are able to draw exact samples from the original, unrelaxed model.
We present a lightweight, a language independent implementation of predicate exchange that modulates the execution of a simulation based model.
Finally, we demonstrate our approach on a number of standard benchmarks and present case studies in inverse graphics, and physiological forecasting.