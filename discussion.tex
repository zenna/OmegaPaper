\section{Discussion}
In this work we expanded the class of predicates that probabilistic models can be conditioned on in practice.

Control flow remains a challenge, and is related to the path explosion problem in program analysis.
Various strategies have been developed \cite{cadar2008exe, sen2005cute} to mitigate this, largely for automated tested.
Future work is to explore the extent to which these concepts can be adapted to the probabilistic domain.

There are several inference strategies other than replica exchange MCMC that could exploit predicate relaxation.
Maximum posterior inference is the most immediate option, which would entail maximizing equation \ref{fpm}.

% Black-box inference methods have gained significant traction due to how general, flexible, a vs grey box

Our approach comes with certain limitations.
Equality conditions on continuous variables indicate sets of zero measure.
This is problematic because the probability of proposing a satisfying state in a Markov chain becomes zero.
In these cases predicate exchange must sample at a minimum temperature strictly greater than zero, which is approximate.
Another limitation occurs if a predicate has branches (e.g., if-then-else statements) such that the execution path taken depends on uncertainty in the model.
Such branches make it more difficult to estimate how close values of variables are to satisfying a predicate.
