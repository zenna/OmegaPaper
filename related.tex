\section{Related Work}

% Likelihood free inference
Demand for likelihood-free inference methods emerged from problems in genetics ecology, where high-fidelity simulations are plentiful but tractable likelihood functions are not.
Approximate Bayesian Computation (ABC) refers to a class of methods \cite{beaumont2002approximate,sisson2007sequential} that sample from posterior distributions by finding simulations that yield data that is sufficiently close to the observed data.
Closeness is determined by a distance function.
Methods which combine ABC Markov Chain Monte Carlo were initialized in \cite{marjoram2003markov} and developed further \cite{wegmann2009efficient}.
Our approach shares target simulation based models, and rely on distance functions.

Predicate exchange targets exact inference without summary statistics, which reduce the dimensionality and to filter out information which is not deemed relevant for the inference of latent variables.
There have been number of approaches to exact inference in implicit models.
Rather than heaveside step function that rejects simulations outside of a distance, \cite{albert2015simulated} parameterizes the distance with a annealing temperature temperature.  If $\alpha$ is declreased sufficently slowly, under some regularity constraints the sequence of targets converges to the true posterior.

\cite{graham2017asymptotically} also targets asymptoically exact inference.
They develop a version of Hamiltonian Monte Carlo at each iteration, uses a quasi-Newton method to solve exactly the constraint that $x = x$.  They perform inference in an unconstrained space and project back to the constrained manifold.
This is of course computationally very expensive.
In addition, it is limited differentiable models conditioned with equality constraints. 

% \cite{Pseudo-Marginal Hamiltonian Monte Carlo}HMC methods cannot be implemented in scenarios where the likelihood function is intractable. However, we have shown here that if we have access to a non-negative unbiased likelihood estimator.
% parameterized by normal random variables then it is possible to derive an algorithm which mimicsthe HMC algorithm having access to the exact likelihood. The resulting pseudo-marginal HMCalgorithm replaces the original intractable gradient of the log-likelihood by the gradient of thelog-likelihood estimator while preserving the target distribution as invariant distributi

% Synthetic Likelihood
-  synthetic likelihood (SL), which uses a multivariate normal approximation of the distribution of a set of summary statistics. 

% PPLS
Probabilistic logics such as ProbLog \cite{richardson2006markov} and Markov logic networks \cite{de2007problog} extend first order logic with probabilistic constructs.
Predicate exchange conditions models on predicates belonging to a Boolean logic, but does not target logic languages in particular.
In fact, it is more suited to stochasticc simlator based models used in more recent probabilistic programming systems~\citep{milch20071, wood2014new,mansinghka2014venture,goodman2008church,carpenter2017stan}.  These systems automatically derive the likelihood function for a rich class of simulation based models, but are subject to the same restrictions that a tractaqble likelihood must exist.

% ?
% Our approach is related to smooth interpretation of programs \citep{chaudhuri2010smooth}

Several continuous logics have been developed to apply model-theoretic tools to metric structures which arise in analysis and geometry.
Continuous logics replace the Boolean structure ${T, F}$, quantifiers $\forall x$ and $\exists x$, and logical connectives with continuous functions,
but vary on the particular continuous structure uses
Predicate relaxation constructs predicates map to continuous structures, but departs from exiting work is primarily in motivation.
Our objective is not to provide a more general logic, but use continuity to make inference more tractable.
Semantically, our approach remains within measure theoretic foundations, which relies on hard predicates to condition.


Probabilistic Similarity Logic \cite{brocheler2012probabilistic,kimmig2012short} uses continuous logics 
Talk about Soft logic



