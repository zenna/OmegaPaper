\section{Related Work}

Demand for likelihood-free inference methods emerged from problems in genetics ecology \cite{}, where high-fidelity simulations are plentiful but tractable likelihood functions are not.
Approximate Bayesian Computation (ABC) methods refers to a class of methods that sample from the posterior distribution over latent variables by finding simulations that result in data that is sufficiently close to the data that was observed.
ABC methods have been successfully used to X, to Y and to Z.
Our approach falls within the ABC paradigm in the sense that (i) we target  simulation based models, and (ii) rely on distance functions.
However, it departs from most ABC methods in motivation
Second, ABC methods rely on summary statistics technically do not use distance functions.
ABC, etc, etc 

Probabilistic programming systems~\citep{milch20071, wood2014new,mansinghka2014venture,goodman2008church,carpenter2017stan} provide a formalism
for declaring probabilistic models and performing inference
in them. Traditional probabilistic programming approaches support model definition by simulation thereby, hiding the underlying probability space. 
One exception to this is Blog \citep{milch20071} 
that defines models in terms of the generating functions.

Our approach is related to smooth interpretation of programs \citep{chaudhuri2010smooth}.

Talk about Soft logic