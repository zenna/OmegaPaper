\section{Related Work}

% Likelihood free inference
Demand for likelihood-free inference emerged in genetics ecology.
Tavar{\'e} et al. \yrcite{tavare1997inferring} computed summary statistics of the output of a simulation, reject an accept-reject for posterior inference.
Weiss et al. \yrcite{weiss1998inference} expanded on this with a tolerance term, such that simulations yielding data sufficiently close to the targets were accepted.
Such posterior samples are therefore approximate.
Approximate Bayesian Computation (ABC) has come to refer to broad class of methods \cite{beaumont2002approximate,sisson2007sequential} in this general regime.
Marjoram et al. \yrcite{marjoram2003markov} simulated Markov Chains according to the prior, but introduced the accept/reject stage to yield approximate posterior samples.
A small tolerance leads to a high rejection rate, whereas a large tolerance results in an unacceptable approximation error.
Among several solutions are dynamically decreasing of tolerance \cite{toni2008approximate}, importance reweighting samples based on distance \cite{wegmann2009efficient}, adapting the tolerance based on distance \cite{del2012adaptive,lenormand2013adaptive}, as well as annealing the tolerance as a temperature parameter \cite{albert2015simulated}.

Predicate exchange targets simulation models and uses distance metrics, but targets exact inference without summary statistics.
In a recent approach with similar motivation, \cite{graham2017asymptotically} develop Hamiltonian Monte Carlo variant, which at each iteration uses a quasi-Newton method to solve exactly the constraint that the data is equal to the output of the generative model.  They perform HMC in an unconstrained space and project back to the constrained manifold.
This is costly -- although the cost is more than accounted for in a reduction of the number of steps -- and limited to differentiable models conditioned with equality. 

% \cite{Pseudo-Marginal Hamiltonian Monte Carlo}HMC methods cannot be implemented in scenarios where the likelihood function is intractable. However, we have shown here that if we have access to a non-negative unbiased likelihood estimator.
% parameterized by normal random variables then it is possible to derive an algorithm which mimicsthe HMC algorithm having access to the exact likelihood. The resulting pseudo-marginal HMCalgorithm replaces the original intractable gradient of the log-likelihood by the gradient of thelog-likelihood estimator while preserving the target distribution as invariant distributi

% PPLS
Probabilistic logics such as ProbLog \cite{richardson2006markov} and Markov logic networks \cite{de2007problog} extend first order logic with probabilistic constructs.
Predicate exchange conditions models on predicates belonging to a Boolean logic, but does not target logic languages in particular.
In fact, it is more suited to stochasticc simlator based models used in more recent probabilistic programming systems~\citep{milch20071, wood2014new,mansinghka2014venture,goodman2008church,carpenter2017stan}.  These systems automatically derive the likelihood function for a rich class of simulation based models, but are subject to the same restrictions that a tractaqble likelihood must exist.

% ?
% Our approach is related to smooth interpretation of programs \citep{chaudhuri2010smooth}

Several continuous logics have been developed to apply model-theoretic tools to metric structures which arise in analysis and geometry.
Continuous logics replace the Boolean structure ${T, F}$, quantifiers $\forall x$ and $\exists x$, and logical connectives with continuous functions,
but vary on the particular continuous structure uses
Predicate relaxation constructs predicates map to continuous structures, but departs from exiting work is primarily in motivation.
Our objective is not to provide a more general logic, but use continuity to make inference more tractable.
Semantically, our approach remains within measure theoretic foundations, which relies on hard predicates to condition.
Probabilistic Similarity Logic \cite{brocheler2012probabilistic,kimmig2012short} uses continuous logics 
Talk about Soft logic



