\section{Related Work}
Talk about likelihood free inference.
ABC, etc, etc 

Probabilistic programming systems~\citep{milch20071, wood2014new,mansinghka2014venture,goodman2008church,carpenter2017stan} provide a formalism
for declaring probabilistic models and performing inference
in them. Traditional probabilistic programming approaches support model definition by simulation thereby, hiding the underlying probability space. 
One exception to this is Blog \citep{milch20071} 
that defines models in terms of the generating functions.

It is hard to compute likelihoods of random variables because they can be arbitrary transformations of fixed randomness. This means for inference in Omega we have to use likelihood-free techniques. These methods can be less efficient than their likelihood-based counterparts. Given the completeness of many probabilistic programming systems, the formalism of Omega can be implemented inside them.

Our approach is related to smooth interpretation of programs \citep{chaudhuri2010smooth}.

Talk about Soft logic