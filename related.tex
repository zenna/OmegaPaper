\section{Related Work}

% Likelihood free inference
Demand for likelihood-free inference emerged in genetics ecology.
Tavar{\'e} et al. \yrcite{tavare1997inferring} 
filtered samples from a stochastic simulator to only those which matched (according to summary statistics) observed data. 
%  to perform one of the earliest forms of likelihood-free inference.
Weiss et al. \yrcite{weiss1998inference} extended this with a tolerance term, so that simulations sufficiently close to the data were accepted.
% Such posterior samples are therefore approximate.
Approximate Bayesian Computation (ABC) now refers methods \cite{beaumont2002approximate,sisson2007sequential} in this general regime.
Marjoram et al. \yrcite{marjoram2003markov} simulated Markov Chains according to the prior, but applied the same summary static based filtering to yield approximate posterior samples.
A small tolerance leads to a high rate of rejected simulations, whereas a large tolerance results in an unacceptable approximation error.
Among several solutions are dynamically decreasing the tolerance \cite{toni2008approximate}, importance reweighting samples based on distance \cite{wegmann2009efficient}, adapting the tolerance based on distance \cite{del2012adaptive,lenormand2013adaptive}, as well as annealing the tolerance as a temperature parameter \cite{albert2015simulated}.

Predicate exchange targets simulation models and uses distance metrics, but targets asymptotically exact inference without summary statistics.
A recent approach \cite{graham2017asymptotically}  with similar objectives develops a Hamiltonian Monte Carlo variant, using a quasi-Newton method during leap-frog integration to exactly solve the observation constraint.
This is limited to differentiable models conditioned with equality.
Predicate exchange is not limited to differentiable models, but can be but is approximate when the condition is of measure zero.

% \cite{Pseudo-Marginal Hamiltonian Monte Carlo}HMC methods cannot be implemented in scenarios where the likelihood function is intractable. However, we have shown here that if we have access to a non-negative unbiased likelihood estimator.
% parameterized by normal random variables then it is possible to derive an algorithm which mimicsthe HMC algorithm having access to the exact likelihood. The resulting pseudo-marginal HMCalgorithm replaces the original intractable gradient of the log-likelihood by the gradient of thelog-likelihood estimator while preserving the target distribution as invariant distributi

% PPLS
Probabilistic logics such as ProbLog \cite{richardson2006markov} and Markov logic networks \cite{de2007problog} allow extend first order logic to declare both models and conditions.
More recent probabilistic programming systems~\citep{milch20071, wood2014new,mansinghka2014venture,goodman2008church,carpenter2017stan} have focused on stochastic simulation, and automatically automatically derive the likelihood function for a rich class of models.


% ?
% Our approach is related to smooth interpretation of programs \citep{chaudhuri2010smooth}

Several continuous \cite{levin2000continuous} and fuzzy \cite{klir1995fuzzy} logics apply model-theoretic tools to metric structures.
Continuous logics replace the Boolean structure $\{T, F\}$, quantifiers $\forall x$ and $\exists x$, and logical connectives with continuous counter-parts.
Predicate relies uses continuous logic only make inference more tractable. Semantically, our approach remains within measure theoretic foundations, which relies on hard predicates to condition.
% Probabilistic Similarity Logic \cite{brocheler2012probabilistic,kimmig2012short} uses continuous logics 
% Talk about Soft logic



