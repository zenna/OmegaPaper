\section{Implementation}\label{implement}

Our approach to inference is not black-box.
It requires a transformation of the model.
This can be realized in a number of ways.
To formalize this we introduce a very simple language for describing probabilistic models.
Following this, we demonstrate how these principle can be incorporated into existing languages.


\subsection{A Minimal Language}{\label{minilang}}


\begin{figure}[t]
	\begin{align*}
		\text { model term }  &  & \enspace m ::=               & e ; \cond f \\
		\text { standard term }  &  & \enspace e ::=               & e ; e \mid v \sim f\\
		\text { standard term }  &  & \enspace f ::=               & p \mid f \textrm{ bop }f \mid\textrm{ op } f \mid \\
		\text { standard term }  &  & \enspace f ::=               & \text{ if } t_1 \text{ then } t_2 \text{ else } t_3 \\
		\text { binary op }      &  & \enspace \textrm{bop} ::=    & + \mid - \mid / \mid * \mid \land \mid \lor \mid > \mid < \mid \\
		\text { unary op }       &  & \enspace \textrm{uop} ::=    & \lnot                                                          \\
		\text { primitive dist } &  & \enspace p ::= \bern(f) \mid & \unif(f, f) \mid N(f, f) \mid                                  \\
	\end{align*}
	\caption{Abstract Syntax}
	\label{syntax}
\end{figure}

Figure \ref{Syntax} describes the abstract syntax of our language.
The language closey resembles statistical notation.
One difference is that conditions are stated at the end of each model in a single statement $\cond$.

Here is an example.

\begin{align*}
	x \sim   & \unif(0, 1)           \\
	y \sim   & \unif(0, 1)           \\
	\cond \; & (x = y) \land (x > 3) \\
\end{align*}

\subsubsection{Semantics}\label{semantics}
\newcommand{\sem}[1]{\llbracket #1 \rrbracket}
Here we define a semantics denotationally.
The denotation $\sem{t}$ of a term $t$ is a value in a semantic domain corresponding to an \omegalang{} type, such as a Boolean, real number, or random variable.
Primitive 


\subsection{Syntactic Predicate Relaxation}

The transformation from the original model to a relaxed model is straight forward.
Algorithm substitutes X accepts as input the abstract syntax

\begin{align*}
	x \sim \unif(0, 1)              \\
	y \sim \unif(0, 1)              \\
	ll \sim x =_s y \land_s x >_s 3 \\
\end{align*}

\subsection{Intercepting the Random Number Generator}\label{rng}

Describe how you can implement this thing by intercepting the random 
of \citep{wingate2011lightweight, milch20071} to MCMC that mutates a database of named random variables: